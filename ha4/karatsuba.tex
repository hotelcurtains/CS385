\documentclass{article}
\usepackage{amsmath}
\usepackage{amsfonts}
\usepackage{mathbbol}
\usepackage{mathtools}
\usepackage[letterpaper,top=1in,bottom=1in,left=1in,right=1in]{geometry}
\usepackage{chngcntr}
\usepackage{amssymb}
\usepackage[verbose]{placeins}

\begin{document}
\raggedright
To find $c = a*b$, where n is the maximum number of digits in $a, b$, B is the base of $a, b$, and $a_0, a_1$ are the least and most $n/2$-digit part of $a$ respectively, and the same goes for $b$ with $b_0, b_1$:
\begin{gather*}
    c_0 = a_0*b_0\\
    c_1 = a_1*b_0 + a_0*b_1 = (a_1 + a_0) * (b_1 + b_0)\\
    c_2 = a_1*b_1\\
    c = a*b = c_2 B^n + (c_1 - c_2 - c_0) B^{n/2} + c_0
\end{gather*}

\section{}
Let's try $2205 * 1132$ (where $n=4$, $B=10$):
\begin{gather*}
    c_0 = 5*32 \\
    c_1 = (22 + 5) * (11 + 32) = 27 * 43\\
    c_2 = 22*11
\end{gather*}
$c_0 = 5 * 32$ (where $n=2$). Since we end up with $n/2 = 1$, we can calculate these single-digit-operand products in the traditional way:
\begin{gather*}
    c_0 = 5*2 = 10\\
    c_1 = (0 + 5) * (3 + 2) = 5 * 5 = 25\\
    c_2 = 0*3 = 0\\
    c = 0*10^2 + (25 - 0 - 10) 10^{2/2} + 10 = 160
\end{gather*}
We'll save $c_0 = 160$ for the final calculation.

$c_1 = 27 * 43$. Again, we're multiplying single digits:
\begin{gather*}
    c_0 = 7*3 = 21\\
    c_1 = (2 + 7) * (4 + 3) = 9 * 7 = 63\\
    c_2 = 2*4 = 8\\
    c = 8 * 10^2 + (63 - 8 - 21) 10^{2/2} + 21 = 1161
\end{gather*}
Which means $c_1 = 1161$.

$c_2 = 22*11$:
\begin{gather*}
    c_0 = 2*1 = 2\\
    c_1 = (2 + 2) * (1 + 1) = 4 * 2 = 8\\
    c_2 = 2*1 = 2\\
    c = 2 * 10^2 + (8 - 2 - 2) 10^{2/2} + 2 = 242
\end{gather*}
Which means $c_2 = 242$.

We can come back to our original c:
\begin{align*}
    c &= a*b = 2205 * 1132\\
      &= c_2 B^n + (c_1 - c_2 - c_0) B^{n/2} + c_0\\
      &= 242 * 10^4 + (1161 - 242 - 160) 10^{4/2} + 160\\
      &= 2496060
\end{align*}
Which leaves us with $2205 * 1132 = 2496060$.

\end{document}