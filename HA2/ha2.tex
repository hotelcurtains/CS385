\documentclass{article}
\usepackage{amsmath}
\usepackage{amsfonts}
\usepackage{mathbbol}
\usepackage{mathtools}
\usepackage[letterpaper,top=1in,bottom=1in,left=1in,right=1in]{geometry}
\usepackage{chngcntr}
\usepackage{amssymb}
\usepackage[verbose]{placeins}
\usepackage{hieroglf}

% sets sections/subsections to be in form 1, 1.a, 2, 2.a, 2.b, etc.
\counterwithin*{equation}{section}
\counterwithin*{equation}{subsection}
\renewcommand{\thesubsection}{\thesection.\alph{subsection}}

% sets up \floor{} and \ceil{} and automatically stretches the symbols to the content
\DeclarePairedDelimiter{\floor}{\lfloor}{\rfloor}
\DeclarePairedDelimiter{\ceil}{\lceil}{\rceil}

% forces correct spacing for "mod" operator in math sections
\renewcommand{\mod}{\text{ mod }}

%allows simpler syntax for logs with long bases and operands
\newcommand{\logb}[2]{{\text{log}_{#1}(#2)}}


% DO:
% p. 67, #4 a, b, c, d, e (1 point each)
% p. 76, #1 a, b, c, d, e (5 points each) 
% p. 76-77, #3 a (5 points), b (5 points)

\title{HA2: Recurrence Relations}
\author{Daniel Detore\\CS 385}
\date{October 2, 2024}

\begin{document}
\maketitle
\raggedright


\section{p. 67, \#4}

\subsection{}
$Mystery(n)$ computes the sum of the squares of all integers from 1 to $n$ inclusive.

\subsection{}
Its basic operation is addition.

\subsection{}
Addition is executed $n$ times for each call of $Mystery(n)$.

\subsection{}
Its efficiency class is $\Theta(n)$.

\subsection{}
Instead of the current strategy, calculating $Mystery(n) = \sum_{i=1}^{n} i^2$ with $\Theta(n)$, we can use my function $\textpmhg{\Hibw}(n) = \frac{n(n+1)(2n+1)}{6}$ (only one calculation no matter the value of $n$) to enter efficiency class $\Theta(1)$.\\
Let's prove this inductively. First, we can establish $\textpmhg{\Hibw}(1) = Mystery(1)$ as $n=1$ is the least value in the domain of $Mystery(n)$:
\begin{align*}
    Mystery(1) &= \textpmhg{\Hibw}(1)\\
    \sum_{i=1}^{1} i^2 &= \frac{1(1+1)(2(1)+1)}{6}\\
    1^2 &= \frac{6}{6}\\
    1 &= 1
\end{align*}
Then we can prove that $\textpmhg{\Hibw}(n) = \textpmhg{\Hibw}(n-1) + n^2$, which means that this formula is doing what it's supposed to do, i.e. summing the squares of the first $n$ positive integers:
\begin{align*}
    \textpmhg{\Hibw}(n) &= \textpmhg{\Hibw}(n-1) + n^2\\
                        &= \frac{(n-1)(n-1+1)(2(n-1)+1)}{6} + \frac{6n^2}{6}\\
                        &= \frac{(n-1)n(2n-1)+ 6n^2}{6}\\
                        &= \frac{2n^3+3n^2+n}{6}\\
                        &= \frac{n(n+1)(2n+1)}{6}
\end{align*}
$\square$

\section{p. 76, \#1}

\subsection{}
Given $x(n) = x(n-1) + 5$ for $n>1$, $x(1) = 0$:
\begin{enumerate}
    \item Replace $n$ by $n-1$: $x(n-1) = x(n-2) + 5$\\
          $\implies$ $x(n) = (x(n-2) + 5) + 5 = x(n-2) + 10$.
    \item Replace $n$ by $n-2$: $x(n-2) = x(n-3) + 5$\\
          $\implies x(n) = (x(n-3) + 5) + 10 = x(n-3) + 15$.
    \item We can generalize to $x(n) = x(n-i) + 5i$.
    \item Initial condition $x(1) = 1 \implies n - i = 1 \implies i = n-1$.
    \item Replace $i$: $x(n) = x(1) + 5(n-1) = 0 + 5(n-1)$.
\end{enumerate}
Solution: $x(n) = 5(n-1)$.

\subsection{}
Given $x(n) = 3x(n-1)$ for $n>1$, $x(1) = 4$:
\begin{enumerate}
    \item $x(n-1) = 3x(n-2)$\\
          $\implies$ $x(n) = 3(3x(n-2)) = 3*3*x(n-2)$.
    \item Replace $n$ by $n-2$: $3x(n-2) = 3x(n-3)$\\
          $\implies x(n) = 3(3(3x(n-3))) = 3*3*3*x(x-3)$.
    \item We can generalize to $x(n) = 3^i * x(n-i)$.
    \item Initial condition $x(1) = 4 \implies n - i = 1 \implies i = n - 1$.
    \item Replace $i$: $x(n) = 3^{n-1} * x(n-(n-1)) = 3^{n-1} * x(1)$.
\end{enumerate}
Solution: $x(n) = 3^{n-1} * 4$.

\subsection{}
Given $x(n) = x(n-1) + n$ for $n>0,$ $x(0) = 0$:
\begin{enumerate}
    \item $x(n-1) = x(n-2) + n-1$\\
          $\implies$ $x(n) = (x(n-2) + n-1) + n = x(n-2) + 2n - 1$.
    \item $x(n-2) = x(n-3) + n - 2$\\
          $\implies x(n) = (x(n-3) + n - 2) + 2n - 1 = x(n-3) + 3n - 3$.
    \item We can generalize to $x(n) = x(n-i) + (i+1)n - i+1$.
    \item Initial condition $x(0) = 0 \implies n - i = 0 \implies i = n$.
    \item Replace $i$: $x(n) = x(n-n) + (n+1)n - (n+1) = x(0) + n^2 + n - n + 1$.
\end{enumerate}
Solution: $x(n) = n^2 + 1$.

\subsection{}
Given $x(n) = x(n/2) + n$ for $n > 1$, $x(1) = 1$:
\begin{enumerate}
      \item Replace $n$ by $2^k$: $x(2^k) = x(2^k/2) + 2^k = x(2^{k-1}) + 2^k$.
      \item Replace $2^k$ by $2^{k-1}$: $x(2^{k-1}) = x(2^{k-1}/2) + 2^{k-1} = x(2^{k-2}) + 2^{k-1}$\\
            $\implies x(2^k) = x(2^{k-2}) + 2^{k-1} + 2^k$.
      \item We can generalize to $x(2^k) = 2^0 + 2^1 + 2^2 + \cdots + 2^k$. This summation is geometric which means $x(2^k) = \frac{2^{k+1}-1}{2-1} = 2^{k+1}-1$.
      \item Initial condition $x(1) = 1 \implies 2^{k-i} = n \implies k = \lg n$. 
      \item Replace $k$: $x(n) = 2^{\lg(n)+1}-1 = 2n-1$.
\end{enumerate}
Solution: $x(n) = 2n - 1$.

\subsection{}
Given $x(n) = x(n/3) + 1$ for $n>1$, $x(1) = 1$:
\begin{enumerate}
    \item Replace $n$ by $3^k$: $x(3^k) = x(3^k/3) + 1 = x(3^{k-1}) + 1$\\
    \item Replace $n$ by $3^{k-1}$: $x(3^{k-1}) = x(3^{k-1}/3) + 1 = x(3^{x-2}) + 1$\\
          $\implies x(3^k) = (x(3^{x-2}) + 1) + 1$.
    \item We can generalize to $x(3^k) = k + 1$.
    \item Inital condition $x(1) = 1 \implies n = 3^k \implies k = \logb{3}{n}$
    \item Replace $k$: $x(n) = \logb{3}{n} + 1$
\end{enumerate}


\section{p. 76-77, \#3}

\subsection{}
Given $S(n) = S(n-1) + n^3$, $S(1) = 1$:
\begin{enumerate}
    \item Replace $n$ by $n-1$: $S(n-1) = S(n-2) + (n-1)^3$\\
          $\implies$ $S(n) = S(n-2) + (n-1)^3 + n^3$.
    \item Replace $n$ by $n-2$: $S(n-2) = S(n-3) + (n-2)^3$\\
          $\implies S(n) = S(n-3) + (n-2)^3 + (n-1)^3 + n^3$.
    \item We can generalize to $S(n) = S(n-i) + (n-i)^3 + (n-i+1)^3 + (n-i+2)^3 + \cdots + n^3 = S(n-i) + \sum_{k=0}^{i-1} (n-k)^3$. 
    \item Initial condition $x(1) = 1 \implies n - i = 1 \implies i = n-1$.
    \item Replace $i$: $S(n) = S(1) + \sum_{k=0}^{n-2} (n-k)^3 = 1 + \sum_{k=0}^{n-2} (n-k)^3$.
\end{enumerate}
Solution: $S(n) = 1 + \sum_{k=0}^{n-2} (n-k)^3$. This means that the algorithm's basic operation is executed $n-1$ times.

\subsection{}
This iterative (nonrecursive) algorithm and the recursive algorithm are the same in terms of running time as they both run at $\Theta(n)$.


\end{document}